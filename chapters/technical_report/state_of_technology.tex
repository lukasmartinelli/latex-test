\chapter{State of Technology}

As of today most web maps are based on raster tiles except a few big providers like Google, Mapbox and Apple.
%\marginpar{The \hyperref[history-of-webmaps]{\emph{History of Webmaps}} chapter provides more details.}
The rest of the industry is now in the transition from raster based maps into vector based maps. For vector based maps the Mapbox Vector Tile Specification\footnote{\url{https://github.com/mapbox/vector-tile-spec/tree/master/1.0.1}} is the most dominant Open Source specification of vector tiles.

\section{Current Vector Data Providers}

While the deployment setup for vector based maps is much simpler than
the traditional raster tile setup the most complex part is still
the creation of vector data which takes a lot of time and care.

\paragraph{Mapbox}

Mapbox provides vector tiles of the entire world as part of its
map hosting technologies since 2012 branded as Mapbox Streets\footnote{\url{https://www.mapbox.com/blog/announcing-mapbox-streets/}}. Mapbox also provides the vector data for buyers of their Mapbox Atlas Server\footnote{\url{https://www.mapbox.com/atlas/}} starting at \$49,000 and distributes quarterly updates for \$10,000 per year .

\paragraph{Mapzen}

Mapzen provides API access to their public vector tiles\footnote{\url{https://mapzen.com/projects/vector-tiles/}}.
%\marginpar{Mapzen vector tiles can also be requested in GeoJSON, TopoJSON and Open Science Map format.}
Mapzen states that access and the platform should remain free and Open Source\footnote{\url{https://mapzen.com/about/}}. Mapzen however
does not give access to the entire raw data and one is bound to
the limitations of the service.

\paragraph{Kartotherian}

The Kartotherian\footnote{\url{https://www.mediawiki.org/wiki/Maps}} project from the Wikimedia Foundation\footnote{\url{https://wikimediafoundation.org/wiki/}} is very similar to this project.
The goal is to provide a free Map service that everyone can use for free.
The process is documented but the data is still bound to the service and not available as download.
The quality of the vector tiles is continuously improved but still lacking very important features.
Kartotherian is a great project which we can contribute to with our improved map data.

\paragraph{Google and Apple}

Apple started using vector tiles in their Apple Maps product in 2012\cite{wiki:apple-maps}  and  Google is using vector tiles since 2013\cite{wiki:google-maps} but they are both not accessible for the general public and use a proprietary format. 

\section{Characteristics}

While the providers open the process they still own the data in order to keep their strategic advantages and promote the use of their products using this data.
The open process is a wonderful step in the right direction, yet it requires great understanding
of the technology and sufficient computing power to actually
execute the workflow of producing vector tiles with worldwide coverage.

\section{Shortcomings}

The providers already solve the problem of producing the vector tiles
and making it accessible. The data however can only be used
via their services.
\newline{}
Being able to download world data and use it in a custom server
infrastructure, offline and without strings attached is a big advantage
and a problem this thesis wants to solve.

\newpage