\chapter{State of Technology}

Currently most web maps are based on raster tiles. Vector tile based maps are
a privilege of a few big providers like Google, Mapbox and Apple until now. The rest of the industry is now in the transition from raster based maps into vector based maps.

For vector based maps the Mapbox Vector Tile Specification\footnote{\url{https://github.com/mapbox/vector-tile-spec/tree/master/1.0.1}} is the most dominant implementation.

Take a look at the \hyperref[history-of-webmaps]{\emph{History of Webmaps}} chapter on (p. \pageref{history-of-webmaps}) which goes more into detail.

\section{Current Vector Data Providers}

While the deployment setup for vector based maps is much simpler than
the traditional raster tile setup the most complex part remains
the creation of vector data which takes alot of time and care.

\paragraph{Mapbox}

Mapbox provides vector tiles of the entire world as part of its
map hosting technologies since 2012 branded as Mapbox Streets\footnote{\url{https://www.mapbox.com/blog/announcing-mapbox-streets/}}. Mapbox also provides the vector data for buyers of their Mapbox Atlas Server\footnote{\url{https://www.mapbox.com/atlas/}} starting at \$49,000 and distributes quarterly updates for \$10,000 per year .

\paragraph{Mapzen}

Mapzen provides API access to their public vector tiles\footnote{\url{https://mapzen.com/projects/vector-tiles/}}.
\marginpar{Mapzen vector tiles can also be requested in GeoJSON, TopoJSON and Open Science Map format}. Mapzen states that access should remain free and Open Source.

Quote:

We build these things and provide them as a service, but these are tools that developers at any level can easily set up and use themselves, for free. We don’t want to lock users into proprietary platforms, relying on black boxes they can’t take apart.

\paragraph{Google}

Google is using vector tiles since 2013 but they are not accessible for
the general public and cannot be licensed as well.

\paragraph{Apple}

Apple is using propretiary vector tiles for their Apple Maps product.

\section{Characteristics}

\section{Shortcomings}

Potential improvements