\chapter{Introduction}

%-------------------------------------------------------

Creating a custom styled OSM map is one of the most common use cases
among cartographers yet it is very difficult to do so. With the new
emerging technology of vector tiles it is possible to allow anyone to
create their custom OSM maps without downloading the entire database and
managing complex infrastructure. The task of this project is to make
this as easy as possible.

%------------------------------------------------------

\section{Vision}

Michal Migurski published on March 15, 2013 a blog
post\footnote{\url{http://mike.teczno.com/notes/postgreslessness-mapnik-vectiles.html}},
in which he described his first attempts to use vector tiles as a source
for the Mapnik tile renderer. In this post he describes the vision of
vector tiles.

Vector tiles only contain geometries and meta data. The visual style can
be applied when the tiles get requested. Since vector tiles do not
contain any information about the style, they are smaller than raster
tiles. This enables high resolution maps, fast map loads and efficient
caching.

Our mission is to bring the power of vector tiles to anyone and provide
the data free and as Open Source project.

%------------------------------------------------------

\section{Targets}\label{targets}

The main target of this thesis is to allow anyone to create their custom
OSM map without managing complex infrastructure. This main target is
split into multiple sub targets, which we defined in the project proposal
at the beginning.

\begin{itemize}
\item
  Deliver a Docker Container to create vector tiles from OSM Data
\item
  Provide the vector tiles for Switzerland
\item
  Provide a Docker Container to serve the vector tiles together with
  custom styles as raster data
\item
  Optional: Vector tiles for the whole world
\end{itemize}

%--------------------------------------------------------

\section{Project procedure}

This project was roughly split into the following steps:

\begin{itemize}
\item
  Inception: Kickoff meeting with Petr Pridal, Stefan Keller and
  definition of project proposal
\item
  Prototyping: At the beginning we spent some time to get to know the
  whole software stack and create a small prototype of every part.
\item
  Evaluation and Requirements: Evaluation of different parts of the
  stack like data import tools, vector tile server. The evaluation of
  this technology as well as additional technologies are part of the
  section technology evaluation.
\item
  Implementation: The creation of the map was an iterative process. We
  started at the lowest zoom level (14) and implemented level by level up
  to zoom level 0. On the way we identified many different problems
  which are further described in the section implementation of the
  project documentation. We documented all of our decisions during the
  implementation process, so that Petr Pridal and Stefan Keller could
  follow every step we took.
\item
  Optimizations: Once we had a first alpha version, we added an
  additional data source (Natural Earth) to further improve the quality
  of the upper zoom levels. During the implementation phase we did a
  refactoring of the project structure and added continuous deployment
  to our setup.
\item
  Rendering: At the end of the project there was a long period for the
  actual rendering of the vector tiles.
\end{itemize}
\newpage