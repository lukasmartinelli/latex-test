\chapter{Vision}\label{vision}

ToDO: Reference Vision of Technical Report

\section{History of Webmaps}
\label{history-of-webmaps}

Web mapping has gone through different technologies and changes in
the recent years. It is important to understand the evolution of web maps to understand why vector tiles are quite a fundamental change in how maps work.

\paragraph{Phase 1: Untiled Static
Maps}

In the beginning WMS servers generated static images for an extract
of the map. Each view of a map requested a unique extract of a map that was generated for this person.


\paragraph{Phase 2: Raster Tiles}

In 2005 Google introduced Google Maps and XYZ 
tiles\footnote{\url{http://wiki.openstreetmap.org/wiki/Slippy_Map}}
which delivered a idempotent raster image for coordinates specified by a
tile index.


\paragraph{Phase 2.5: Raster Tiles with Vector
Overlays}

To provide a level of interactivity tools like
Leaflet\footnote{\url{http://leafletjs.com/}} support rendering vector
data like SVG on top of a raster base map.

In order to support fully interactive maps 

\paragraph{Phase 2.75: Raster Tiles from Vector
Tiles}

For backwards compatibility and faster serving of raster tiles vector
tiles where introduced to avoid querying a database.

\paragraph{Phase 3: Vector Tiles}

Vector tiles are delivered directly to the browser and rendered by Web
GL based clients.


Improving the use of vector data in web mapping is often shown as the next challenge
of web mapping \cite[p.~88]{gaffuri2012toward} 
