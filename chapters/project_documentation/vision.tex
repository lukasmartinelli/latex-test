\chapter{Vision}\label{vision}

The vision of our project is described in part 1, section~\ref{part1_vision}. This chapter should give a bit of background on where the idea of vector tiles came from.

\section{History of Webmaps}
\label{history-of-webmaps}

Web mapping has gone through different technological changes in recent years. It is important to understand the evolution of web maps to understand why vector tiles are quite a fundamental change in how maps work.

\paragraph{Phase 1: Untiled Static
Maps}

In the beginning WMS servers generated static images for an extract
of the map. Each view of a map requested a unique extract of a map that was generated for this person.

\paragraph{Phase 2: Raster Tiles}

In 2005 Google introduced Google Maps and XYZ 
tiles\footnote{\url{http://wiki.openstreetmap.org/wiki/Slippy_Map}}
which delivered a idempotent raster image for coordinates specified by a
tile index.

\paragraph{Phase 2.5: Raster Tiles with Vector
Overlays}

To provide a level of interactivity, tools like
Leaflet\footnote{\url{http://leafletjs.com/}} support rendering vector
data like SVG on top of a raster based maps.

\paragraph{Phase 2.75: Raster Tiles from Vector
Tiles}

For backwards compatibility and faster serving of raster tiles vector
tiles where introduced to avoid querying a database.

\paragraph{Phase 3: Vector Tiles}

Vector tiles are delivered directly to the browser and rendered by Web
GL based clients.
\newline{}
Improving the use of vector data in web mapping is often shown as the next challenge
of web mapping \cite[p.~88]{gaffuri2012toward} 
