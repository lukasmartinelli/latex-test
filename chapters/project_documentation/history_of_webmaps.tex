\chapter{History of Webmaps}
\label{history-of-webmaps}

Web mapping has gone through different technologies and changes in
the recent years. It is important to understand the evolution of web maps to understand why vector tiles are quite a fundamental change in how maps work.

\paragraph{Phase 1: Untiled Static
Maps}

In the beginning WMS servers generated static images for an extract
of the map. Each view of a map requested a unique extract of a map that was generated for this person.


\paragraph{Phase 2: Raster Tiles}

In 2005 Google introduced Google Maps and XYZ 
tiles\footnote{\url{http://wiki.openstreetmap.org/wiki/Slippy_Map}}
which delivered a idempotent raster image for coordinates specified by a
tile index.


\paragraph{Phase 2.5: Raster Tiles with Vector
Overlays}

To provide a level of interactivity tools like
Leaflet\footnote{\url{http://leafletjs.com/}} support rendering vector
data like SVG on top of a raster base map.

In order to support fully interactive maps 

\paragraph{Phase 2.75: Raster Tiles from Vector
Tiles}

For backwards compatibility and faster serving of raster tiles vector
tiles where introduced to avoid querying a database.

\paragraph{Phase 3: Vector Tiles}

Vector tiles are delivered directly to the browser and rendered by Web
GL based clients.


Improving the use of vector data in web mapping is often shown as the next challenge
of web mapping \cite[p.~88]{gaffuri2012toward} 

\subsubsection{Vector Tile Formats}

\paragraph{Mapbox Vector Tiles}

When Mapbox introduced it's geography tool Mapbox Studio in 2013 they
created the \emph{Mapbox Vector Tiles Specification}
\footnote{\url{https://github.com/mapbox/vector-tile-spec}} which is
implemented by a variety of tools and clients
\footnote{\url{https://github.com/mapbox/awesome-vector-tiles}}
including \emph{Mapbox GL JS}, \emph{Open Layers 3}, \emph{Leaflet},
\emph{Mapzen Tangram} and Esri
\footnote{\url{https://www.mapbox.com/blog/vector-tile-adoption/}} in
the future.

\subsubsection{Geopackage}

The \emph{GeoPackage Encoding Standard} is the OGC counterpart to the
\emph{Mapbox Vector Tiles Specification} which was introduced later and
is supported by QGIS, ESRI and GDAL.

\subsubsection{Google Maps}

Google Maps is using vector tiles since 2010 under the hood and was the
first provider implementing this. Styling is limited and the format
propretiary.

\subsubsection{Mapbox}

In 2013 Mapbox introduced Mapbox Studio a geography tool working with
vector tiles.

\subsubsection{Mapzen}

Was machen andere / welche ähnlichen Arbeiten gibt es zum Thema? Was
kann von anderen verwendet werden?

\subsubsection{Vector Tile Providers}\label{vector-tile-providers}

\paragraph{Mapbox}

\paragraph{Kartotherian}

\paragraph{Mapzen}

Diese Einleitung soll für den Ingenieur irgendeiner Fachrichtung
verständlich sein.

Sie stellt die Aufgabe in einen grösseren Zusammenhang und liefert eine
genaue Beschreibung der Problemstellung.

Allfällige Vorarbeiten oder ähnlich gelagerte Arbeiten werden
diskutiert.

Theoretische Grundlagen sind nur so weit auszuarbeiten, als dies für die
Lösung der Aufgabe nötig ist (keine Lehrbücher schreiben).

Die Erkenntnisse aus den theoretischen Untersuchungen sind wenn immer
möglich direkt mit der Problemlösung zu verknüpfen.