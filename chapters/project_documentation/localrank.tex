\chapter{Relative Importance}
\label{localrank}

To reduce label density on lower zoom levels but still contain all data in e.g. zoom level 14 the \texttt{localrank} attribute indiciates how
important a label is compared to the labels in its neighbourhood.

\section{Calculating Rank}

And then create the local rank for each tile in a 128 px grid when returning the POIs.

In the best case scenario one would create a function that
ranks each individual point of interest. In this case we only ranked the most important features.

\subsection{Order Features by their Types}

\begin{sqlcode}
CREATE OR REPLACE FUNCTION localrank_poi(type VARCHAR) RETURNS INTEGER
AS $$
BEGIN
  RETURN CASE
    WHEN type IN ('station', 'subway_entrance', 'park',
                  'cemetery', 'bank', 'supermarket', 'car',
                  'library', 'university', 'college', 'police',
                  'townhall', 'courthouse') THEN 2
    WHEN type IN ('nature_reserve', 'garden', 'public_building') THEN 3
    WHEN type IN ('stadium') THEN 90
    WHEN type IN ('hospital') THEN 100
    WHEN type IN ('zoo') THEN 200
    WHEN type IN ('university', 'school', 'college', 'kindergarten') THEN 300
    WHEN type IN ('supermarket', 'department_store') THEN 400
    WHEN type IN ('nature_reserve', 'swimming_area') THEN 500
    WHEN type IN ('attraction') THEN 600
    ELSE 1000
  END;
END;
$$ LANGUAGE plpgsql IMMUTABLE;
\end{sqlcode}


\subsection{Calculate Rank of Features}

The rank is calculated across a grid of 128 pixels. Our most
important features from the \texttt{localrank\_poi} function will
also be the most relevant POIs.

\begin{sqlcode}
SELECT
  geometry,
  rank() OVER (PARTITION BY LabelGrid(geometry, 128 * !pixel_width!)
               ORDER BY localrank_poi(type) ASC) AS localrank,
FROM osm_poi
\end{sqlcode}