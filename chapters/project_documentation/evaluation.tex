\chapter{Technology Evaluation}\label{technology_evaluation}

\section{Spatial Database}\label{spatial_database}

The OSM wiki\footnote{\url{http://wiki.openstreetmap.org/wiki/Databases_and_data_access_APIs\#Database_Schemas}} recommends PostgreSQL with the PostGIS extension for processing the data. There also exists a method\footnote{\url{https://github.com/systemed/tilemaker}} that circumvents using a database and directly transform OSM data into vector tiles but this does not scale for global vector tile coverage and does not support mixing additional data into the vector tiles.

\\
\begin{tcolorbox}[arc=0mm,boxrule=1pt,title=Decision]\label{spatial_db_decision}
PostGIS was the only viable choice due to superb tooling support for OSM data and
advanced spatial query capabilities.
\end{tcolorbox}

\section{OSM Import Tool}\label{osm_import_tool}
As import tool the OSM community recommends imposm\footnote{\url{http://imposm.org/}} or osm2pgsql\footnote{\url{http://wiki.openstreetmap.org/wiki/Osm2pgsql}}.
In this section the two tools are compared with each other.

\subsection{Criterias}\label{criterias}

\paragraph{Speed} 
In order to iterate fast and be able to change the data style frequently
it is important that the import tool is reasonably fast and is able
to import the OSM planet file in one single day.

\paragraph{Customized Schema}
Customizing a schema to already split up features into separate tables
makes querying more performant and easier to do.

\paragraph{Diff Updates}
It must be possible to apply the Planet diffs \footnote{\url{http://wiki.openstreetmap.org/wiki/Planet.osm/diffs}} 
to continuously update the database with newer data.
\\
If a import is very fast it is also possible to simply reimport the latest
planet dump.

\paragraph{Existing Data Style Projects}
In order to get started it is helpful to have a lot of query examples
from other data style projects available.

\subsection{Evaluation Matrix}\label{evaluation_matrix}

\begin{table}[H]
\centering
    \begin{tabular}{llll}
    \hline
    Criteria         & Weight & imposm & osm2pgsql \\
    \hline
    Speed             & 0,3    & 8      & 5         \\
    Customized Schema & 0,4    & 7      & 4         \\
    Diff Updates      & 0,2    & 6      & 8         \\
    Existing Material & 0,1    & 6      & 10        \\
    \hline
    \textbf{Weighted Score} & 1      & 7      & 5,7       \\
    \end{tabular}
    \caption{Evaluation matrix of imposm vs osm2pgsql}
\end{table}


\subsection{osm2pgsql}\label{osm2pgsql_importer}

osm2pgsql\footnote{\url{http://wiki.openstreetmap.org/wiki/Osm2pgsql}} is the
most commonly used import tool for processing raw OpenStreetMap data into PostGIS.
The import schema is also called osm2pgsql and defines a very
simple schema(line, point, polygon and
roads)\footnote{\url{http://wiki.openstreetmap.org/wiki/Osm2pgsql/schema}}.
This results in very large tables, so it is recommended to create good
indices. Osm2pgsql supports updating of the database, if the values have
been stored as hstore. The schema can be adapted via the import style \footnote{\url{http://wiki.openstreetmap.org/wiki/Osm2pgsql\#Import_style}}
but most projects use the default style\footnote{\url{https://github.com/openstreetmap/osm2pgsql/blob/master/default.style}} provided by osm2pgsql.

\subsubsection{imposm3}\label{imposm-importer}

Imposm is an import tool for OSM data, it is not a schema. But it
defines a default
schema\footnote{\url{http://imposm.org/docs/imposm/latest/database_schema.html}},
which could possibly be changed by provinding a custom mapping file. An
advantage of the default schema is that it groups data thematically into
tables. Which results in smaller tables and simpler queries. Imposm 3
supports updating the database from OSM diff
files\footnote{\url{http://imposm.org/docs/imposm3/latest/tutorial.html\#diff}}

\\ 
\begin{tcolorbox}[arc=0mm,boxrule=1pt,title=Decision]\label{osm_import_tool_decision}
For our use case it is important, that the import is efficent and that
the import tool supports updating based on OSM diff files. Imposm 3 is
faster than osm2pgsql and supports updatability. So we decided to take
imposm for importing.
\end{tcolorbox}

\section{Vector Tile Format}\label{vector-tile-formats}

Vector tiles is quite a broad term. If we talk about vector tiles in thesis we usually mean Mapbox vector tiles which is a custom open specification how vector tiles should be structured.

\paragraph{Mapbox Vector Tiles}

When Mapbox introduced it's geography tool Mapbox Studio in 2013 they
created the \emph{Mapbox Vector Tiles Specification}
\footnote{\url{https://github.com/mapbox/vector-tile-spec}} which is
implemented by a variety of tools and clients
\footnote{\url{https://github.com/mapbox/awesome-vector-tiles}}
including \emph{Mapbox GL JS}, \emph{Open Layers 3}, \emph{Leaflet},
\emph{Mapzen Tangram} and Esri
\footnote{\url{https://www.mapbox.com/blog/vector-tile-adoption/}} in
the future.

\paragraph{Geopackage}

The \emph{GeoPackage Encoding Standard} is the OGC counterpart to the
\emph{Mapbox Vector Tiles Specification} which was introduced later and
is supported by QGIS, ESRI and GDAL.

\paragraph{Google Maps}

Google Maps is using vector tiles since 2010 under the hood and was the
first provider implementing this. Styling is limited and the format
proprietary.

\\
\begin{tcolorbox}[arc=0mm,boxrule=1pt,title=Decision]\label{vector_tile_spec_impl_decision}
Because one of the main requirements of this project was to provide Mapbox Streets compatible vector tiles and Mapbox provides very good tools to handle vector tiles, we had almost no other choice than going with Mapbox's implementation of vector tiles.
\end{tcolorbox}

\section{Vector Tile Server}\label{vector_tile_server}

Next to the vector tiles for Switzerland, the second deliverable is a basic vector tile server. The goal is that a non technical person can get started quickly with the custom vector tiles.\\

Unlike the choice for a spatial database or OSM import tool
there is no typical setup method of a raster tile server using vector tiles under the hood. Most people in the Open Source community build their own specific tile server setup.

\subsection{Tessera}\label{tessera}

Tessera\footnote{\url{https://github.com/mojodna/tessera}} is a Node.js\footnote{\url{https://nodejs.org/en/}} webserver, which s using Mapbox tilelive\footnote{\url{https://github.com/mapbox/tilelive}} modules to read vector tiles and generates raster tiles.

\begin{figure}[H]
\centering
  \includegraphics[scale=0.6]{images/tessera_node_bindings.png}
  \caption{Architecture Diagramm Tessera}
\end{figure}

Load tests were performed to measure server performance with tessera.
\\
\textbf{Infrastructure}: AWS EC2 t2.micro instance(1 GB Memory / 1 Core)
\textbf{Task}: Zooming in from zoom level 10 to 22 \\
\textbf{Conditions}: 50 concurrent users \\

A sample result of the load test is shown below.

\begin{bashcode}
---- Global Information --------------------------------------------------------
> request count                                      10350 (OK=10348  KO=2     )
> min response time                                     50 (OK=50     KO=60010 )
> max response time                                  60335 (OK=56103  KO=60335 )
> mean response time                                  2138 (OK=2127   KO=60172 )
> std deviation                                       3023 (OK=2914   KO=162   )
> response time 50th percentile                       1115 (OK=1114   KO=60172 )
> response time 75th percentile                       3127 (OK=3125   KO=60253 )
> mean requests/sec                                 75.919 (OK=75.904 KO=0.015 )
---- Response Time Distribution ------------------------------------------------
> t < 800 ms                                          4545 ( 44%)
> 800 ms < t < 1200 ms                                 756 (  7%)
> t > 1200 ms                                         5047 ( 49%)
> failed                                                 2 (  0%)
---- Errors --------------------------------------------------------------------
> java.util.concurrent.TimeoutException: Request timed out to ec      2 (100.0%)
2-52-30-184-45.eu-west-1.compute.amazonaws.com/52.30.184.45:80...
\end{bashcode}

With 50 concurrent users tessera was still able to respond to all requests in less than 1200 ms.

\subsection{OpenStreetMap "Standard" Tile Server}\label{osm_standard_tile_server}

A proven set up for generating raster tiles directly from PostgreSQL with Mapnik consists of an Apache webserver and a custom Apache module mod\_tile\footnote{\url{http://wiki.openstreetmap.org/wiki/Mod_tile}}. This approach was around before vector tiles were proposed.

\begin{figure}[H]
\centering
  \includegraphics[scale=0.6]{images/mod_tile.png}
  \caption{Architecture Diagramm OSM Standard tile server}
\end{figure}

It is theoretically possible to implement a raster tile server
on top of Mapnik and renderd using C++. Instead of using a spatial database as source for the rendered raster tiles a datasource binding would need to be implemented to read the vector tiles and hand them over to the Mapnik renderer. 

\\
\begin{tcolorbox}[arc=0mm,boxrule=1pt,title=Decision]\label{tile_server_decision}
The plan was to perform the load test on each version of the vector tile server and then make the decision based on the results. Due to the fact, that the test results of tessera where good enough for the thesis use case and it didn't take much time to implement, we decided to use tessera as raster tile server.
If somebody needs a high-performance tile server, one should probably think about the second variant.
\end{tcolorbox}