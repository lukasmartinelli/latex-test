\chapter{Technology Evaluation}

\section{Spatial Database}

PostGIS was the only viable choice due to superbe tooling support.

\section{OSM Import Tool}\label{osm-import-tool}
The OSM community
recommends\footnote{\url{http://wiki.openstreetmap.org/wiki/Databases_and_data_access_APIs}}
PostgreSQL with the Postgis extension and imposm or osm2pgsql as import tool.

\subsection{Criterias}

\paragraph{Speed} 
In order to iterate fast and be able to change the data style frequently
it is important that the import tool is reasonably fast and is able
to import the OSM planet file in one single day.

\paragraph{Customized Schema}
Customizing a schema to already split up features into separate tables
makes querying more performant and easier to do.

\paragraph{Diff Updates}
It must be possible to applying Planet diffs \footnote{\url{http://wiki.openstreetmap.org/wiki/Planet.osm/diffs}} 
to continuously update the database with newer data.

If a import is very fast it is also possible to simply reimport the latest
planet dump.

\paragraph{Existing Data Style Projects}
In order to get started it is helpful to have alot of query example
from other data style projects available.

\subsection{Evaluation Matrix}

\begin{center}
    \begin{tabular}{llll}
    \hline
    Criteria         & Weight & imposm & osm2pgsql \\
    \hline
    Speed             & 0,3    & 8      & 5         \\
    Customized Schema & 0,4    & 7      & 4         \\
    Diff Updates      & 0,2    & 6      & 8         \\
    Existing Material & 0,1    & 6      & 10        \\
    \hline
    \textbf{Weighted Score} & 1      & 7      & 5,7       \\
    \end{tabular}
\end{center}


\subsection{osm2pgsql}\label{osm2pgsql-importer}

osm2pgsql \footnote{\url{http://wiki.openstreetmap.org/wiki/Osm2pgsql}} is the
most commonly used import tool for processing raw OpenStreetMap data into PostGIS.
The import schema is also called osm2pgsql and defines a very
simple schema(line, point, polygon and
roads)\footnote{\url{http://wiki.openstreetmap.org/wiki/Osm2pgsql/schema}}.
This results in very large tables, so it is recommended to create good
indices. Osm2pgsql supports updating of the database, if the values have
been stored as hstore.

The schema can be adapted via the import style \footnote{\url{http://wiki.openstreetmap.org/wiki/Osm2pgsql\#Import_style}}
but most projects use the default style\footnote{\url{https://github.com/openstreetmap/osm2pgsql/blob/master/default.style}} provided by osm2pgsql.

For our use case it is important, that the import is efficent and that
the import tool supports updating based on OSM diff files. Imposm 3 is
faster than osm2pgsql and supports updatability. So we decided to take
imposm for importing.

\subsection{Imposm 3 }\label{imposm-importer}

Imposm is an import tool for osm data, it is not a schema. But it
defines a default
schema\footnote{\url{http://imposm.org/docs/imposm/latest/database_schema.html}},
which could possibly be changed by provinding a custom mapping file. An
advantage of the default schema is that it groups data thematically into
tables. Which results in smaller tables and simpler queries. Imposm 3
supports updating the database from OSM diff
files\footnote{\url{http://imposm.org/docs/imposm3/latest/tutorial.html\#diff}}

TODO: formulate decision

\section{Vector Tile Format}\label{vector-tile-formats}

Vector tiles is quite a broad term. If we talk about vector tiles in thesis we usually mean Mapbox vector tiles which is just a custom open specification how vector tiles should be structured.

\paragraph{Mapbox Vector Tiles}

When Mapbox introduced it's geography tool Mapbox Studio in 2013 they
created the \emph{Mapbox Vector Tiles Specification}
\footnote{\url{https://github.com/mapbox/vector-tile-spec}} which is
implemented by a variety of tools and clients
\footnote{\url{https://github.com/mapbox/awesome-vector-tiles}}
including \emph{Mapbox GL JS}, \emph{Open Layers 3}, \emph{Leaflet},
\emph{Mapzen Tangram} and Esri
\footnote{\url{https://www.mapbox.com/blog/vector-tile-adoption/}} in
the future.

\paragraph{Geopackage}

The \emph{GeoPackage Encoding Standard} is the OGC counterpart to the
\emph{Mapbox Vector Tiles Specification} which was introduced later and
is supported by QGIS, ESRI and GDAL.

\paragraph{Google Maps}

Google Maps is using vector tiles since 2010 under the hood and was the
first provider implementing this. Styling is limited and the format
proprietary.

TODO: formulate decision

\section{Vector Tile Server}

\subsection{Tessera}

\subsection{Apache}