\chapter{Requirements Specification}\label{requirements_specification}

This chapter describes the requirements for the vector tiles and the vector tile server.

\section{User Characteristics}\label{user_characteristics}

There are three user groups interested in this project:

\begin{itemize}
\item
  \textbf{Map Designer}: A technically versed person using Windows or
  OSX with knowledge of GIS software but not necessarily of it's
  inner technical workings.
\item
  \textbf{System Administrator}: The person which needs to host the
  published maps.
\item
  \textbf{Geographic Institutes or other companies}: Institutions which would like to provide their own geospatial data in vector tile format. 
\end{itemize}

\section{User Stories}\label{user_stories}
\paragraph{Serverside Rendering}
The map designer brings his new map design to the system administrator and tells him to host the map. The system administrator can download the vector tiles of the osm2vectortiles website and serve the map with the help of a vector tile server.

\paragraph{Clientside Rendering}
The map designer brings his new map design to the system administrator and tells him to host the map. The system administrator does not want to provide a big server that can handle the heavy rendering work. So he decides to use Mapbox GL\cite{rs_1_mapbox.com_2015} which offloads the rendering work to the client. In order to use Mapbox GL he needs to statically serve the vector tiles. 

\paragraph{Own vector tiles}
Institutions which provide geospatial data would like to follow the trend of vector tiles and provide their data in vector tile format. These institutions can use this project for this matter. They can easily modify this project to use it with their data. 

\section{Non Functional Requirements}\label{non_functional_requirements}

The non functional requirements are the key to success of this project. If the following requirements can be fulfilled, the specified users will be able to benefit form our project.

\paragraph{Usability}
The vector tile server must be usable with Kitematic\cite{rs_2_kitematic_2015}. Kitematic is an easy to use user interface for docker. 

\paragraph{Learnability}

Map designers should not have to learn how to use the command line in
order to use Docker.

\paragraph{Repeatable}

Generating OSM vector tiles must be possible in a weekly interval because OSM
updates regularly.

\paragraph{Performance}

The tileserver must handle 10 concurrent users per second.

\paragraph{Compatibility}

The vector tiles must contain all feature sets Mapbox Streets contains. If full compatibility with Mapbox Streets\cite{22_mapbox.com_2015} can be guaranteed all Mapbox visual styles can be used with our vector tiles.

\paragraph{Vector Tile Size}

The size of a single vector tile should not be greater than 500 KB.
