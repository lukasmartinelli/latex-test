\chapter{Requirements Specification}
\label{requirements-specification}

\section{User Characteristics}
\label{user-characteristics}

There are two user groups interested in this project:

\begin{itemize}
\item
  \textbf{Map Designer}: A technically versed person using Windows or
  OSX which has knowledge of GIS software but not necessarily of it's
  inner technical workings.
\item
  \textbf{System Administrator}: The person who needs to host the
  published maps.
\end{itemize}

\section{Actors and Stakeholders}
\label{actors-and-stakeholders}

\section{Use Cases}
\label{use-cases}

NOTE: I actually found use cases to be not so useful in SE2 project. But
somehow this project is not as ``imaginable'' as a web application.
Perhaps short uses cases are still a good idea.

But perhaps this should be done more in the form of user stories (a bit
more informal).

\section{Non Functional Requirements}
\label{non-functional-requirements}

The most important NFR in this scenario is usability. If we cannot get
usability right no one will use our solution.

\paragraph{Usability}

\paragraph{Learnability}

Map designers should not have to learn how to use the command line in
order to use Docker.

\paragraph{Repeatable}

Generating OSM vector tiles must be possible any time because OSM
updates regularly.

\paragraph{Performance}

The tileserver needs to be reasonably fast but not highly performant.

TODO: Tell numbers