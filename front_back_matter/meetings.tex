\chapter*{Meeting Protocols}

\section*{7th September 2015}

\subsection*{Participants}
\begin{itemize}
  \item Petr Pridal
  \item Stefan Keller
  \item Lukas Martinelli
  \item Manuel Roth
\end{itemize}


\subsection*{Talking points}

\begin{itemize}
  \item Get to know each other
  \item Project goal
  \item Architecture of Docker containers
  \item Definition of tasks and milestones
\end{itemize}

\section*{16th September 2015}

\subsection*{Participants}

\begin{itemize}
  \item Stefan Keller
  \item Lukas Martinelli
  \item Manuel Roth
\end{itemize}


\subsection*{Talking points}

\begin{itemize}
  \item Language choice of thesis
  \item Similarity of documentation to SE2 project
  \item Role of Petr Pidal
  \item Mechanics of MBTiles
\end{itemize}

\subsection*{Protocol}
\begin{itemize}
  \item Create a project proposal
  \item Inform ourselves about Meta Tiles
  \item Few prerendered tiles can cover most of the used map
  \item Before each meeting, send email with done, difficulties and plan for next week to Stefan Keller
  \item Thesis of prior years can be found at HSR eprints
\end{itemize}


\subsection*{Questions and Answers}

- Can the thesis be written in English?
\textit{Yes, that's up to you. Other groups did write their project documentation in English and the actual thesis in German.}

- Should the documentation be made similar to SE2?
\textit{Yes, but only the parts with make sense to your project. Usually, the thesis is divided into two parts, the first part should be like an article of a computer magazine (C't). The second part all formal documents (requirements analysis, domain analysis, use cases)}

- What's Mr.Pridal's role? - Does Mr.Pridal affect the evaluation?
\textit{Mr.Pridal can basically be viewed as an industrial partner. I will seek his opinion at the end of the project. If there should be differences between you and him, I will have the final word.}

\section*{25th September 2015}

\subsection*{Participants}

\begin{itemize}
  \item Stefan Keller
  \item Lukas Martinelli
  \item Manuel Roth
\end{itemize}


\subsection*{Done}

\begin{itemize}
  \item Went through OSM Workflow ( OSM Source -> Import OSM Data -> Postgis -> Source Project -> Export mbtiles -> Serving mbtiles)
\end{itemize}

\subsection*{Difficulties}

\begin{itemize}
  \item It was not clear, if we could take an existing source project or if we had to make our own.
\end{itemize}

\subsection*{Plan}

\begin{itemize}
  \item Documenting OSM Workflow
  \item Comparing osm2pgsql and imposm3
  \item Meeting with Petr Pridal on Monday
\end{itemize}

\subsection*{Talking points}

\begin{itemize}
  \item Imposm3 or osm2pgsql vor import: many people in the OSM community are used to osm2pgsql and its default schema (point, line, polygon, road). The criteria for choosing the import tool, should be efficient and have the possibility to update data with OSM diff files.
  \item Thesis should compare GeoPackage and MBTiles 
  \item Term feature set is more accurate than layer
\end{itemize}

\section*{28th September 2015}

\subsection*{Participants}

\begin{itemize}
  \item Petr Pridal
  \item Lukas Martinelli
  \item Manuel Roth
\end{itemize}


\subsection*{Done}

\begin{itemize}
  \item Reached alpha milestone(first version of components)
\end{itemize}

\subsection*{Difficulties}

\begin{itemize}
  \item We need to define system requirements for the containers
  \item On small machines(> 500 RAM) null pointer exceptions while importing with imposm3
  \item What is the purpose of the debug viewer
\end{itemize}

\subsection*{Plan}

\begin{itemize}
  \item Next meeting with Petr: Monday 5pm, 19th of October 2015
  \item Decision of which rendering and tile serving stack should be choosen (based on performance tests with each stack)
\end{itemize}

\subsection*{Talking points}

\begin{itemize}
  \item Can current software stack meet our requirements?
  \subitem Requirements osm2vectortile stack: Render whoule planet in reasonable amount of time(less than one month)
  \subitem Requirements tileserver: not yet defined
  \item Performance:
  \subitem Compare Vector tile rendering stack with wikimedia's
  \subsubitem Import only necessary data for rendering
  \subsubitem Compare your source project with wikimedia's
  \subsubitem Why did wikimedia fork mapbox's osm-brigt style?
  \subitem Tileserver: Compare performance of node + nginx with apache + renderd
  \item The debug viewer is needed, if you want to view the mbtiles without Mapbox studio
  \subitem Petr would be interested to integrate thiw viewer into tileserver-php
  \item Rename tileserver to vtileserver for clarity
\end{itemize}

\section*{2th October 2015}

\subsection*{Participants}

\begin{itemize}
  \item Stefan Keller
  \item Manuel Roth
\end{itemize}


\subsection*{Done}

\begin{itemize}
  \item Documented OSM worklow
  \item Meeting with Petr
  \item OSM Planet file imported
  \item Passed alpha milestone
\end{itemize}

\subsection*{Difficulties}

\begin{itemize}
  \item OSX postgres volume mounting problem
\end{itemize}

\subsection*{Plan}

\begin{itemize}
  \item Compare Wikimedia OSM stack with current stack (performance)
  \item Decision of which rendering and tile serving stack should be choosen (based on performance tests with each stack)
\end{itemize}

\subsection*{Questions and Answers}

- How many hours does each of us has to invest into the thesis?
\textit{Every team member has to invest at least 240h into the thesis (8 ECTS * 30h)}

- On which server can we generate the mbtiles and how up-to-date should they be? (weekly updates)
\textit{You can use the IFS server for the mbtiles generation. If there are problems with the server, please contact Mirko Stocker. There are no special requirements for how current the mbtiles have to be}

\subsection*{Talking points}
\begin{itemize}
  \item Stefan Keller creates the project proposal and sends it for review back to us. On next weeks meeting we are going to sign the project proposal.
  \item Stefan Keller's contact person at wikimedia maps: Tim alias Kolossos (tim@alder-digital.de)
\end{itemize}

\section*{16th October 2015}

\subsection*{Participants}

\begin{itemize}
  \item Stefan Keller
  \item Lukas Martinelli
  \item Manuel Roth
\end{itemize}


\subsection*{Done}

\begin{itemize}
  \item Structure of thesis defined
  \item Debug Viewer created
  \item Import performance analysis
  \item Performance scenarios for vector tile server defined (Gatling.io)
\end{itemize}

\subsection*{Difficulties}

\begin{itemize}
  \item Mapnik with vector tiles as source
  \item SQL queries are bottleneck of mbtiles creation
\end{itemize}

\subsection*{Plan}

\begin{itemize}
  \item Performance tests with tessera and apache tile server
  \item Mbtiles for whole switzerland
\end{itemize}

\section*{30th October 2015}

\subsection*{Participants}

\begin{itemize}
  \item Stefan Keller
  \item Lukas Martinelli
  \item Manuel Roth
\end{itemize}

\subsection*{Done}

\begin{itemize}
  \item Started with data style
  \item Created tool to analyze mapbox streets vector tiles
  \item Started working on zoom level 14
  \item Added lots of data to data style
\end{itemize}

\subsection*{Difficulties}

\begin{itemize}
  \item Find differences to our vector tiles
\end{itemize}

\subsection*{Plan}

\begin{itemize}
  \item Identify missing items/layers
  \item Update data style
\end{itemize}


\section*{13th November 2015}

\subsection*{Participants}

\begin{itemize}
  \item Stefan Keller
  \item Lukas Martinelli
  \item Manuel Roth
\end{itemize}

\subsection*{Done}

\begin{itemize}
  \item Work on zoom level 14 is almost finished
  \item Work on zoom level 10 to 13 has started
  \item Chapter technical report started
\end{itemize}

\subsection*{Plan}

\begin{itemize}
  \item Finish zoom level 10 to 13
  \item Work from lowest zoom level(14) to smallest zoom level(0)
  \item Finish chapter technical report
\end{itemize}

\section*{27th November 2015}

\subsection*{Participants}

\begin{itemize}
  \item Stefan Keller
  \item Lukas Martinelli
  \item Manuel Roth
\end{itemize}

\subsection*{Done}

\begin{itemize}
  \item Work on zoom level 10 to 13 is finished
  \item Work on zoom level 0 to 10 has started with additional data source
  \item Chapter project documentation started
\end{itemize}

\subsection*{Difficulties}

\begin{itemize}
  \item For upper zoom levels Natural Earth data set is needed (0-8)
\end{itemize}

\subsection*{Plan}

\begin{itemize}
  \item Finish zoom level 0 to 10
  \item Finish chapter project documentation
\end{itemize}

\section*{11th December 2015}

\subsection*{Participants}

\begin{itemize}
  \item Stefan Keller
  \item Lukas Martinelli
  \item Manuel Roth
\end{itemize}

\subsection*{Done}

\begin{itemize}
  \item Finished work on data style
  \item Chapter technical report, project documentation finished
  \item Draft of chapter project management created
\end{itemize}

\subsection*{Plan}

\begin{itemize}
  \item Finish chapter project management
  \item Revise thesis with feedback 
  \item Create extracts for project website
  \item Create user guide
\end{itemize}

\section*{18th December 2015}

\subsection*{Participants}

\begin{itemize}
  \item Stefan Keller
  \item Lukas Martinelli
  \item Manuel Roth
\end{itemize}

\subsection*{Done}

\begin{itemize}
  \item Fished project thesis
  \item Launched project website
  \item Launched project with press release (social media, blogs)
  \item Created cd, poster, added thesis to HSR abstract tool
\end{itemize}

\newpage{}